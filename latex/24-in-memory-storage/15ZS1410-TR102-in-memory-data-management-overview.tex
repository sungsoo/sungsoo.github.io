
\documentclass[twocolumn]{article}
\usepackage{mathpazo}
\usepackage{microtype}
\usepackage{times}
\usepackage{titlesec} % 1
\usepackage[colorlinks = true,
            linkcolor = blue,
            urlcolor  = blue,
            citecolor = blue,
            anchorcolor = blue]{hyperref}
%\usepackage{sectsty} % "제 1 절" ...

 %%%%%%%%%%%%%%%%%%%%%%%%%%%%%%%%%%%%%%%%%%%%%%%%%%%%%%%%%%%%%%%%%%%%%%%%%%%%%
 %                              My Commands
\newcommand{\bi}{\begin{itemize}}
\newcommand{\ei}{\end{itemize}}
\newcommand{\be}{\begin{enumerate}}
\newcommand{\ee}{\end{enumerate}}
\newcommand{\ii}{\item}
\newtheorem{Def}{Definition}
\newtheorem{Lem}{Lemma}

%\usepackage{algorithm}
%\usepackage{algorithmicx}
%\usepackage{algpseudocode}

\usepackage{algpseudocode,algorithm,algorithmicx}
\newcommand*\DNA{\textsc{dna}}

\newcommand*\Let[2]{\State #1 $\gets$ #2}
\algrenewcommand\algorithmicrequire{\textbf{Input:}}
\algrenewcommand\algorithmicensure{\textbf{Output:}}


\usepackage{graphicx}
\graphicspath{%
        {converted_graphics/}
        {./images/}
}

\usepackage{color}
\usepackage{xcolor}
\usepackage{listings}
\usepackage{caption}
\DeclareCaptionFont{white}{\color{white}}
\DeclareCaptionFormat{listing}{\colorbox{gray}{\parbox{\textwidth}{#1#2#3}}}
\captionsetup[lstlisting]{format=listing,labelfont=white,textfont=white}
\usepackage{verbatimbox}

\usepackage[hangul,nonfrench,finemath]{kotex}
    
\setlength\textwidth{7in} 
\setlength\textheight{9.5in} 
\setlength\oddsidemargin{-0.25in} 
\setlength\topmargin{-0.25in} 
\setlength\headheight{0in} 
\setlength\headsep{0in} 
%\setlength\columnsep{5pt}
\sloppy 
 
\begin{document}

\title{
\vspace{-0.5in}\rule{\textwidth}{2pt}
\begin{tabular}{ll}\begin{minipage}{4.75in}\vspace{6px}
\noindent\large {\it KIWI Project}@Data Management Research Section\\
\vspace{-12px}\\
\noindent\LARGE ETRI\qquad  \large Technical Report 15ZS1410-TR-102
\end{minipage}&\begin{minipage}{2in}\vspace{6px}\small
218 Gajeong-ro, Yuseong-gu\\
Daejeon, 305-700, South Korea\\
http:/$\!$/www.etri.re.kr/\\
http:/$\!$/sungsoo.github.com/\quad 
\end{minipage}\end{tabular}
\rule{\textwidth}{2pt}\vspace{0.25in}
\LARGE \bf 인-메모리 데이터 관리 개요 \\
\large Overview of In-Memory Data Management
}

\date{}

\author{
{\bf Sung-Soo Kim}\\
\it{sungsoo@etri.re.kr}
}

\maketitle

\begin{abstract}
인-메모리 컴퓨팅 (In-Memory Computing; IMC)이란 어플리케이션을 구동하는 컴퓨터의 메인 메모리에 DB 데이터와 같은 주요 데이터를 저장하고 처리하는 컴퓨팅 기술을 말한다.
인-메모리 컴퓨팅은 초기에는 증권사의 실시간 트레이딩, 통신사의 로그인 세션 관리 등 빠른 처리가 필수적인 OLTP 데이터 처리에 주로 사용됐으나, 최근에는 분석용 DBMS, 시각화 기반 데이터 탐색 도구, 하둡 기반 빅데이터 분석 등 분석 어플리케이션을 위한 기반 기술로 자리 잡았다. 

KIWI는  DAG (Directed Acyclic Graph) 기반 엔진을 이용하여 질의를 처리하고 있다. 질의 처리 과정에서 클러스터 각 노드에 중간 처리결과를 로컬 디스크에 저장한 후, 최종결과를 취합하는 과정을 거친다. 성능 개선을 위해, 이러한 중간 결과를 인-메모리 분산저장소인 타키온(Tachyon)에 저장하여 취합하는 방식을 고려하고 있다. 
이와 관련하여 본 기술문서에서는 인-메모리 데이터 관리의 설계 철학, 타키온 구현에 적용된 기술들을 분석한다.
\end{abstract}

%\begin{algorithm}[H]
%  \caption{Counting mismatches between two packed \DNA strings 
%  \label{alg:packed-dna-hamming}}
%  \begin{algorithmic}[1]
%    \Require{$x$ and $y$ are packed \DNA strings of equal length $n$}
%    \Ensure{$\delta$ is return value}
%    \Statex
%    \Function{Distance}{$x, y$}
%      \Let{$z$}{$x \oplus y$} \Comment{$\oplus$: bitwise exclusive-or}
%      \Let{$\delta$}{$0$}
%      \For{$i \gets 1 \textrm{ to } n$}
%        \If{$z_i \neq 0$}
%          \Let{$\delta$}{$\delta + 1$}
%        \EndIf
%      \EndFor
%      \State \Return{$\delta$}
%    \EndFunction
%  \end{algorithmic}
%\end{algorithm}

\section{Introduction}
Jim Gray's insight that ``\textit{Memory is the new disk, disk is the new tape}'' is becoming true today. 
In the last decade, multi-core processors and the availability of large amounts of main memory at
plummeting cost are creating new breakthroughs, making it viable to
build in-memory systems where a significant part, if not the entirety,
of the database fits in memory \cite{Hao:2015}.

Database systems have been evolving over the last few decades, mainly
driven by advances in hardware, availability of a large amount of data,
collection of data at an unprecedented rate, emerging applications and
so on. The landscape of data management systems is increasingly
fragmented based on application domains (i.e., applications relying on
relational data, graph-based data, stream data).

\section{Tachyon Overview}
Memory is the key to fast big data processing. This has been realized by many, and frameworks such as Spark and Shark already leverage memory performance. As data sets continue to grow, storage is increasingly becoming a critical bottleneck in many workloads.
Tachyon is a memory-centric fault-tolerant distributed storage system, which enables reliable file sharing at memory-speed across cluster frameworks such as Spark and MapReduce. 
Tachyon is Hadoop compatible. Existing Spark and MapReduce programs can run on top of it without any code changes. Tachyon is the default off-heap option in Spark, which means that RDDs can automatically be stored inside Tachyon to make Spark more resilient and avoid GC overheads. 

\bibliographystyle{abbrv}
\bibliography{sqlonhadoop-2015-10}

\end{document}

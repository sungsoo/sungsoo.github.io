% This is a simple LaTex sample document that gives a submission format
%   for IEEE PAMI-TC conference submissions.  Use at your own risk.
% Make two column format for LaTex 2e.\
\documentclass[11pt,twocolumn]{article} %,twocolumn
\usepackage{times,amsmath,amsfonts}

% Use following instead for LaTex 2.09 (may need some other mods as well).
%\documentstyle[times,twocolumn]{article}
\usepackage[dvips]{graphicx,graphics}
% Set dimensions of columns, gap between columns, and paragraph indent
\setlength{\textheight}{10.5in} \setlength{\textwidth}{7.6in}
%\setlength{\columnsep}{0.3125in} \setlength{\topmargin}{0in}
\setlength{\headheight}{0in} \setlength{\headsep}{-1in}
\setlength{\parindent}{1pc}
\setlength{\oddsidemargin}{-0.6in}  % Centers text.
\setlength{\evensidemargin}{-0.5in}

% Add the period after section numbers.  Adjust spacing.
\newcommand{\Section}[1]{\vspace{-5pt}\section{\hskip -1em.~~#1}\vspace{-3pt}}
\newcommand{\SubSection}[1]{\vspace{-2pt}\subsection{\hskip -1em.~~#1}
        \vspace{-3pt}}
\newcommand{\bqn}{\begin{eqnarray*}}
\newcommand{\eqn}{\end{eqnarray*}}
\newcommand{\bq}{\begin{eqnarray}}
\newcommand{\eq}{\end{eqnarray}}

\begin{document}

% Make title bold and 14 pt font (Latex default is non-bold, 16pt)
\title{Stat 471: Lecture 27\\
Expectation-Maximization Algorithm III.}
% For single author (just remove % characters)
\author{Moo K. Chung\\
mchung@stat.wisc.edu}
% For two authors (default example)
\maketitle \thispagestyle{empty}
\begin{enumerate}

\item Two component mixture model. We may assume that the observations $y_i$
comes from two distributions $f_1(y)$ and $f_2(y)$ such that 
$$f(y) = pf_1(y) = (1-p)f_2(y)$$


\end{enumerate}
\end{document}

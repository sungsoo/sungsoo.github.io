\section{Beta Process Models For Segmentation}
The Beta-Bernoulli model is a Bayesian non-parametric approach to assign 
a set of observations to a set of features [cite].
One of key aspects of this approach is it allows for clustering with \emph{mult-tenancy}; 
that is observations may be members of multiple clusters.
This approach has been applied the segmentation of dynamical trajectories [cite], however,
has not yet been applied to identify kinematic primitives.

The kinematic case poses a variety of new challenges.
First, we need to explictly make the model invariant to rotations.
The next challenge is that kinematic primitives are more sensitive to
time scales and thus we need to be careful with how we model the windowing operations
and how those windows are allowed to interact with other properties such as rotation invariance.

The standard recipe for Bayesian approaches is to define a generative model; a plausible parameterized statistical model
that generates the observed data.
Then conditioned on the observations infer the expected model.

\textbf{SK: The notation below is inconsistent with the stuff I added}
\subsection{Generative Model For Demonstration-Trajectory Relationship}
We first look at the case for fixed $K$ number of motion primitives.
In this case, we assume the demontration are created by the following
generative model for each demonstration $i$:
\begin{enumerate}
\item Draw $T_{i}\sim Poisson(\gamma)$
\item Draw $\theta\sim Beta(\alpha)$
\item Draw $R\sim N(I,\sigma I)$
\item For $1..t..T_{i}$

\begin{enumerate}
\item For 1..k..K

\begin{enumerate}
\item Choose $z_{ik}\sim Ber(\theta)$ 
\end{enumerate}
\item $\tau_{i}[t]=R(\sum_{k}z_{ik}p_{k})+\epsilon$
\item $\epsilon\sim N(0,\sigma I)$
\end{enumerate}
\end{enumerate}
Now, if we allow $K\rightarrow\infty$, that is a potentially infinite
number of primitives, this model becomes a Beta Process model:
\begin{enumerate}
\item Draw $T_{i}\sim Poisson(\gamma)$
\item Draw $B\sim BP(1,B_{0})$
\item Draw $R\sim N(R_{0},\sigma I)$
\item For $1..t..T_{i}$

\begin{enumerate}
\item $Z\sim BeP(B)$ 
\item $\tau_{i}[t]=R(\sum_{k}^{\infty}Z_{ik}p_{k})+\epsilon$ 
\item $\epsilon\sim N(0,\sigma I)$
\end{enumerate}
\end{enumerate}
Due to the isotropic nature of the generative model, this can be solved
using a variant of BP-Means proposed by Broderic, Kulis, and Jordan
2012 {[}cite{]}. \{TODO: A relaxation is needed to make the translation
invariance work\}


\subsection{Parameter Inference For $\mathbb{P}$}

As in {[}cite{]}, let $1...K^{+}$ index the current set of primitives
$\mathbb{P}$.
\begin{enumerate}
\item For all demonstrations

\begin{enumerate}
\item Calculate expected $R$: via orthonormally constrained least squares
(as in Mahler et al.) fit a rigid transformation to all $\tau\in x_{i}$,$p\in\mathbb{P}$
and take the mean.
\item For all trajectories within a demonstration

\begin{enumerate}
\item Choose subset of primitives that sum to $\tau$
\item Add current trajectory to set of primitives, if it improves the objective
by more than $\lambda^{2}$ then include it else drop it.
\end{enumerate}
\item Update set of primitives via least squares.\end{enumerate}
\end{enumerate}
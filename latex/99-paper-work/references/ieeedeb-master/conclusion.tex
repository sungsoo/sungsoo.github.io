\section{Conclusion}
An important challenge in data analytics is presence of dirty data
in the form of missing, duplicate, incorrect or inconsistent values.
Data analysts report that data cleaning remains one of the most time
consuming steps in the analysis process, and data cleaning can require
a significant amount of developer effort in writing software or rules
to fix the corruption. 
SampleClean studies the integration of Sample-based Approximate Query Processing and data cleaning; to provide analysts a tradeoff between cleaning the entire dataset and avoiding cleaning altogether.
To the best of our knowledge, this is the first work to marry data cleaning with sampling-based query processing.
While sampling introduces approximation error, the data cleaning mitigates errors in query results.
This idea opened up a number of new research opportunities, and we applied the same principles to other domains such as Materialized View Maintenance and Machine Learning.

\vspace{0.5em}

\textbf{\scriptsize We would like to thank Mark Wegman whose ideas helped inspire SampleClean project.
This research would not have been possible without collaboration
with Daniel Haas and Juan Sanchez.
We would also like to acknowledge Kai Zeng, Ben Recht, and Animesh Garg for their input, feedback, and advice throughout the course of this research.
This research is supported in part by NSF CISE Expeditions Award CCF-1139158, DOE Award SN10040 DE-SC0012463, and DARPA XData Award FA8750-12-2-0331, and gifts from Amazon Web Services, Google, IBM, SAP, The Thomas and Stacey Siebel Foundation, Adatao, Adobe, Apple, Inc., Blue Goji, Bosch, Cisco, Cray, Cloudera, EMC2, Ericsson, Facebook, Guavus, HP, Huawei, Informatica, Intel, Microsoft, NetApp, Pivotal, Samsung, Schlumberger, Splunk, Virdata and VMware.}
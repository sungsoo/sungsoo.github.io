%% -*- UTF-8 -*- LaTeX -*-
%% 최영한, 수학 논문의 정리류를 쓰는 요령
%%
%%   원본은 hfont와 ksmeproc을 이용한 문서.
%%   dhucs와 amsproc 문서로 수정함.
%%   본문 내용은 일체 변경하지 않았고 코드 구현만을 부분적으로 수정하였음.
%%       touched by Karnes Kim.
%%
\RequirePackage{setspace}
%
\documentclass[a4paper,10pt]{amsproc}
\usepackage{amssymb}

%%% load dhucs.
\usepackage[nonfrench]{dhucs}
\usepackage{dhucs-ucshyper}
\hypersetup{%
   colorlinks=true
}
\usepackage{memhfixc}
\setInterHangulSkip{0pt}
\let\gt\sffamily
%%%% fonts.
\usepackage[T1]{fontenc}
%%% 폰트 설정은 호환성을 위하여 별도의 스타일로 한다.
%%% krfnt.sty 파일이 없으면 은글꼴 기본 글꼴을 사용함.
\IfFileExists{krfnt.sty}{\usepackage{krfnt}}{}

%% line spread
\linespread{1.3}

%% names.
%\renewcommand\partname{}
\renewcommand\refname{참 고 문 헌}
\renewcommand\abstractname{\gt 요약문}

%% local macros
\newcommand\amsthmstyle{\texttt{amsthm.sty}}
\newenvironment{headnote}%
     {\small\itshape\noindent\ignorespaces}%
     {\vskip2\baselineskip\ignorespaces}
\newcommand\env[1]{\texttt{#1}}

\newsavebox\mybox
\newlength\hunindentlen
\def\gethunindentlen#1{%
 \sbox{\mybox}{\textbf{#1.}}%
 \setlength\hunindentlen{\wd\mybox}%
 \addtolength\hunindentlen{5mm}%
}

\newcommand\cntrdot{\raisebox{.2ex}{\;\bfseries\textperiodcentered\;}}

%%% footnotes.
%%% setspace에 의한 footnote의 재정의는 amsproc에서는
%%% 효력이 없다. 그러므로 다음과 같이 이 명령을 수정해주어야 한다.
%%% 만약 footnote의 행간을 재조정할 생각이 아니라면 아래 정의는
%%% 하지 않아도 무방함.
\AtBeginDocument{\setlength\footnotesep{20pt}}
\makeatletter
\long\def\@footnotetext#1{%
  \insert\footins{%
    \normalfont\footnotesize
    \interlinepenalty\interfootnotelinepenalty
    \splittopskip\footnotesep \splitmaxdepth \dp\strutbox
    \floatingpenalty\@MM \hsize\columnwidth
    \@parboxrestore \parindent\normalparindent \sloppy
    \protected@edef\@currentlabel{%
      \csname p@footnote\endcsname\@thefnmark}%
  \begingroup\setstretch{1.0}% <----------------
    \@makefntext{%
      \rule\z@\footnotesep\ignorespaces#1\unskip\strut\par}%
  \endgroup}% <---------------
}

%%% section. ksmeproc.cls는 \section이 \sc로 찍히도록 해두고 있다.
%%% 아무튼 문서 외양의 호환성을 위해서 ksmeproc.cls의 정의를 가져왔음.
\def\section{\@startsection{section}{1}%
  \z@{1.5\linespacing\@plus\linespacing}{1.2\linespacing}%%1.2 1.5
  {\large\sffamily\scshape\centering}}           %\sc\gt\centering
\makeatother

%%%%%%%%% dhucs에서 amsproc을 쓰기 위한 설정은 다음 한 줄뿐임.
\renewcommand\uppercasenonmath[1]{}
%%%%%%%%%%%%%%%%%%%%%%%%%%%%%%%%%%%%%%%%%%%%%%%%%%%%%%%%

%% title과 author
\title{수학 논문의 정리류(Theorem-like)를 쓰는 요령}
\author[최 영 한]{최 영 한\;(한국과학기술원)}
\address{대전광역시 유성구 구성동 373-1 한국과학기술원
         자연과학부 우편번호 305-701}
\email{yhchoe@kaist.ac.kr}

%%%%%%%%%%%%%%%%%%%%%%%%%%%%%%%%%%%%%%%%%%%%%%%%%%%%%%%%
%%%%% 문서 시작 
%%%%%%%%%%%%%%%%%%%%%%%%%%%%%%%%%%%%%%%%%%%%%%%%%%%%%%%%
\begin{document}
\begin{headnote}
  이 글은 2001년 4월 21일 동의대학교에서 있은 ``대한수학회 2001년도 봄
  연구발표회''에서 발표한 글(최영한 \cite{ch02} 참조)을 바탕으로 최근의
  정보를 보완한 것이다.
\end{headnote}

%% abstract가 \maketitle 보다 먼저 와야 한다.
\begin{abstract}
  \textbf{Theorem}, \textbf{Lemma}, \textbf{Defintion},
  \textbf{Example}, \textbf{Note}, \textit{Remark},
  \textit{Acknowledgment}, $\cdots$ 등의 표제어를 앞머리(Header)로 두는
  문단을 선언적 문단(Pro\-cla\-ma\-tion)이라 한다.  선언적 문단의 스타일로
  Mittelbach~[\textit{An extenssion of the \LaTeX~theorem
  environment}, 2000, p.~3]는 ``theorem style'' 이라하여
  \begin{quote}
     \env{plain}, \env{break}, \env{marginbreak},
     \env{changebreak}, \env{change}, \env{margin}
  \end{quote}
  의 여섯 가지로 분류하였다. 한편 \AmS-\LaTeX 에서는 위에
  열거한 여섯 가지 중 \env{plain} 환경만 채택하고,
  \env{definition} 환경과 \env{remark} 환경을 첨가하여
  ``정리류(Theorem-like) 스타일''이라고 하고
  \begin{quote}
   \env{plain}, \env{definition}, \env{remark}
  \end{quote}
  의 세 가지만을 쓰고 있다. 전자는 문단 모양과
  앞머리(Header)의 위치에 따라서 구분한 것이고, 후자는 글꼴(앞머리의
  글꼴과 본체의 글꼴)에 따라 구분한 것이다.  후자를 Mittelbach의
  분류대로 한다면 모두 plain에 속할 것이다.

  이 글에서는 영문으로 된 순수 및 응용 수학의 논문을 준비할 때
  \LaTeX{} 패키지와 \AmS-\LaTeX{} 패키지의 \amsthmstyle{} 속에
  내장되어 있는 정리류(Theorem-like) 환경을 활용하여 선언적
  문단(Proclamation)을 효과적(읽기 쉽고 보기 좋게)으로 늘어 놓는
  방법에 대해서 알아 보자. 또 \AmS-\LaTeX 의 정리류(Theorem-like)
  환경을 쓰지 않고 \TeX{} 파일을 만들 때 겪게되는 어려움과 이 논문을
  학술지의 게재 논문으로 편집\cntrdot 조판할 때 나타나는
  고충을 알아 본다.
\end{abstract}

%--------------------------------------------------------------------
\maketitle
%%% 별도로 \markboth할 필요가 있다면 다음과 같이 할 것.
%\markboth{최 \ \ 영 \ \ 한}%
%         {순수 및 응용 수학 논문의 정리류(Theorem-like) 쓰는 요령}
%--------------------------------------------------------------------

\section{선언적 문단(Proclamation)의 환경}                             %sec1

순수 및 응용 수학의 논문에는 선언적 문단(Proclamation)\footnote{\label{a1}%각주1
  아직 우리말 번역이 정착되어 있지 않다. ``주장'', ``선언'' 등으로 번역할
  수 있겠으나 필자는 ``선언적 문단''이라 번역하여 보았다.}%
이라 일컫는 ``\textbf{Definition~1}, \textbf{Proposition 2}, \textbf{Theorem 3.1},
\textbf{Corollory 3.2}, \textbf{Fermat's Last Theorem}, \textbf{Lemma
  A}, \textbf{Example}, \textit{Remark}\,'' 등의 표제어를
앞머리(Header)로 두는 문단이 많이 있다. 선언적 문단이 하나도 없는 글은
학술 논문이라고 볼 수 없기 때문에 대게 게재 여부의 심사 대상에서조차 제외한다.  %p2

이런 선언적 문단의 앞머리에는 독특하게 번호를 붙이기도 하고 경우에
따라서는 붙이지 않기도 한다. 번호를 붙이고, 또 붙이지 않는 방법에
대해서는 2절에서 다루기로 하고 여기서는 선언적 문단의 환경에 대해서
알아보자.

Mittelbach~\cite[p.~3]{mi}는 선언적 문단의 스타일을 ``theorem style''
이라하고
\begin{equation*}\label{eq:1}
  \texttt{plain},\;\;\texttt{break},\;\;\texttt{marginbreak},\;\;\texttt{changebreak},
  \texttt{change},\;\;\texttt{margin}        \tag{$\ast$}
\end{equation*}
의 여섯 가지를 들고 있다. 한편 \AmS-\LaTeX 에서는 위
(\ref{eq:1})에 열거한 여섯 가지 중 \env{plain} 환경만 채택하고,
\env{definition} 환경과 \env{remark} 환경을 별도로 만들어
``정리류(Theorem-like) 스타일''이라 하고
%\smallskip
\begin{equation*}\label{eq:2}
 \env{plain},\;\;\env{definition},\;\;\env{remark}     \tag{$\ast\ast$}
\end{equation*}
%\smallskip
\noindent 의 세 가지만을 쓰고 있다. (\ref{eq:1})\는 문단 모양과
앞머리(Header)의 위치에 따라 구분한 것이고, (\ref{eq:2})\는
글꼴(앞머리와 본문의 글꼴)에 따라 구분한 것이다. (\ref{eq:2})\를
(\ref{eq:1})의 분류대로 한다면 모두 plain에 속할 것이다.

\env{plain} 환경에서 선언적 문단의 앞머리는 볼드체로 되고,
본체(body text)는 이탤릭체로 된다.\footnote{\label{a2}% %각주2
  참고로 한국수학교육학회지 시리즈 B: <순수 및 응용 수학>과
  Kyungpook Math. J. 등에서는 \env{plain} 환경을 써야 하는 선언적
  문단(\textbf{Theorem}, \textbf{Lemma}, \textbf{Corollory},
  \textbf{Proposition}, $\cdots$)은 \AmS-\LaTeX 에서 설정한 대로
  앞머리는 볼드체로, 본체는 이탤릭체로 쓰고 있다. 한편 대한수학회에서
  발행하는 세 학술지와 충청수학회지는 앞머리를 \textsc{Small Capital}로
  본체를 \textsl{기울림체} (\textsl{Slant})로 쓰고 있다. 그러나 국제적으로는
  앞머리를 \textsc{Small Capital}로 본체를 이탤릭체로 하는 곳이 훨씬
  많다.}
한편 \env{definition} 환경은 앞머리는 볼드체이고, 본문은
로마체이다. 그리고 \env{remark} 환경은 앞머리는 이탤릭체이고, 본문은
로마체이다. 물론 수식을 포함한 수학적 표현은 어디에서나 수학
환경(Mathematics Environment)에 따르기 때문에 이 원칙을 따르지 않는다.

\env{plain} 환경은 \AmS-\LaTeX 이 아닌 그냥 \LaTeX 패키지에서 이미
정의되어 있으므로 굳이 \AmS-\LaTeX 패키지를 쓰지 않아도 된다. 그러나
\env{definition} 환경과 \env{remark} 환경은
\AmS-\LaTeX{} 패키지(더욱 정확히는 \amsthmstyle )속에 정의 되어
있고 그냥 \LaTeX{} 패키지 속에는 정의되어 있지 않기 때문에
\AmS-\LaTeX{} 패키지를 불러야 이들 환경을 쓸 수 있다.\footnote{\label{a3}% %각주3
  \AmS-\LaTeX{} 패키지를 자동으로 불러 오지 않는 클래스(예:
  \texttt{article.cls}, \texttt{book.cls})를 썼을 때
  \amsthmstyle 를 불러오지 않기 때문에 \env{definition} 환경과
  \env{remark} 환경을 쓰려면 Preamble (\TeX{} 파일의 첫 머리부터
  \texttt{\textbackslash begin\{document\}}의 바로 앞까지)에
  \begin{displaymath}\label{eq:fn}
    \backslash\texttt{usepackage\{amsmath\}}\quad  \mbox{또는} \quad
    \backslash\texttt{usepackage\{amsthm\}}    \tag{$\ast\!\ast\!\ast$}
  \end{displaymath}
  와 같이 \amsthmstyle{} 또는 \amsthmstyle 을 불러야 한다.
  Preamble에서 \amsthmstyle 을 부르면 \amsthmstyle 은 필요할
  때 자동적으로 \amsthmstyle 을 부른다. 물론 \amsthmstyle 만
  부를 수도 있다.\\
  \indent 한국수학교육학회지 시리즈 B: <순수 및 응용 수학>의
  \TeX{} 파일의 맨 첫줄에 보면
  \begin{quote}
  \texttt{\textbackslash documentclass\{ksme-b\}}{\%}수식의 번호를 왼 쪽에 둘 때
  \end{quote}
  또는
  \begin{quote}
  \texttt{\textbackslash documentclass[reqno]\{ksme-b\}}{\%}수식의 번호를 오른 쪽에 둘 때
  \end{quote}
  와 같이 \texttt{ksme-b.cls} 파일을 쓰고 있음을 알 수 있다. 그런데
  이 \texttt{ksme-b.cls} 파일은 자동적으로
  \AmS-\LaTeX{} 패키지(정확하게는 \amsthmstyle )를 불러 오기
  때문에 \AmS-\LaTeX 의 정리류 환경을 쓰기 위하여 별도로
  (\ref{eq:fn})\와 같이 \amsthmstyle  또는 \amsthmstyle 를 부르지 않아도 된다.}

\AmS-\LaTeX 의 세 가지 정리류 환경은 ``MS 워드''에서 ``유형''에
해당한다. 이들 세 유형에 속하는 정리류를 열거하면 다음과 같다.

\begin{description}
\item[\env{plain}] \textbf{Theorem}, \textbf{Lemma},
  \textbf{Corollory}, \textbf{Proposition}, \textbf{Conjecture},
  \textbf{Criterion}, \textbf{Algorithm}, $\cdots$
\item[\env{definition}] \textbf{Definition}, \textbf{Condition},
  \textbf{Problem}, \textbf{Example}, $\cdots$
\item[\env{remark}] \textit{Remark}, \textit{Note},
  \textit{Notation}, \textit{Claim}, \textit{Summary}, \textit{Acknowledgment}, 
  \textit{Case}, \textit{Conclusion}, $\cdots$
\end{description}

\section{번호 붙이기와 번호 붙이지 않기}                               %sec2

정리류 환경을 따른 선언적 문단의 앞머리는 별도로 번호를 원하지 않는다는
표시를 하지않는 한 모두 자동적으로 일련 번호가 매겨진다.  가령
% \begin{quote}
%   \hskip-13mm\textbf{예 1.}\hskip5mm\textbackslash {\tt
%     newtheorem\{defn\}\{Definition\}}\\
%   \textbackslash \texttt{newtheorem\{thm\}\{Theorem\}}\\
%   \textbackslash \texttt{newtheorem\{lem\}\{Lemma\}}\\
%   \textbackslash \texttt{newtheorem\{prop\}\{Proposition\}}\\
%   \textbackslash \texttt{newtheorem\{rem\}\{Remark\}}
% \end{quote}
\begin{quote}
\noindent\gethunindentlen{예 1}
\hspace*{-\hunindentlen}\usebox{\mybox}\hspace{4mm}\verb|\newtheorem{defn}{Definition}| \\
            \verb|\newtheorem{thm}{Theorem}| \\
            \verb|\newtheorem{lem}{Lemma}| \\
            \verb|\newtheorem{prop}{Proposition}| \\
            \verb|\newtheorem{rem}{Remark}|
\end{quote}
와 같이 Preamble 에 정리류 환경을 선택하였다면 이 논문은 \textbf{Definition} 은
\textbf{Definition} 대로 \textbf{Theorem} 은
\textbf{Theorem} 대로 \textbf{Lemma} 는 \textbf{Lemma} 대로
\textbf{Proposition} 은 {\bf Proposition} 대로 그리고 \textit{Remark}
는 \textit{Remark} 대로 각각 ``일련 번호 매기기''를 한다. 그런데 만약
% \begin{quote}
%   \hskip-13mm\textbf{예 2.}\hskip5mm\textbackslash {\tt
%     newtheorem\{defn\}\{Definition\}}\\
%   \textbackslash \texttt{newtheorem\{thm\}\{Theorem\}}\\
%   \textbackslash \texttt{newtheorem\{lem\}[thm]\{Lemma\}}\\
%   \textbackslash \texttt{newtheorem\{prop\}[thm]\{Proposition\}}\\
%   \textbackslash \texttt{newtheorem\{rem\}\{Remark\}}
% \end{quote}
\begin{quote}
\noindent\gethunindentlen{예 2}
\hspace*{-\hunindentlen}\usebox{\mybox}\hspace{4mm}\verb|\newtheorem{defn}{Definition}| \\
   \verb|\newtheorem{thm}{Theorem}| \\
   \verb|\newtheorem{lem}{Lemma}| \\
   \verb|\newtheorem{prop}{Proposition}| \\
   \verb|\newtheorem{rem}{Remark}|
\end{quote}
와 같이 환경을 선택하였다면 이 논문은 \textbf{Theorem},
\textbf{Lemma}, \textbf{Proposition} 은 모두 자동으로 함께 ``일련 번호
매기기''를 하지만 \textbf{Definition} 과 \textit{Remark} 은 각각 따로
``일련 번호 매기기''를 한다.

그런데 \env{plain} 환경은 이미 \LaTeX{} 속에 default 로 정의되어 있기
때문에 구태여
% \begin{quote}
%      \texttt{\textbackslash theoremstyle\{plain\}}
% \end{quote}
\begin{verbatim}
  \theoremstyle{plain}
\end{verbatim}
과 같이 명령어를 주지 않아도 되지만 \env{definition} 환경 또는
\env{remark} 환경을 따르는 선언적 문단을 만들려면 반드시 Preamble
에서 \texttt{\textbackslash newtheorem\{\;\}} 의 \{\;\} 앞쪽에 각각
% \begin{quote}
%      \texttt{\textbackslash theoremstyle\{definition\}\\
%           \textbackslash theoremstyle\{remark\}}
% \end{quote}
\begin{verbatim}
  \theoremstyle{definition}
  \theoremstyle{remark}
\end{verbatim}
을 넣어 어떤 환경을 쓸 것인지 명시하여야 한다. 따라서 예 1, 예 2와 같이
\env{definition} 환경 또는 \env{remark} 환경을 선택하지 않았다면
모든 선언적 문단은 \env{plain} 환경에 따른다.%
\footnote{\label{a4}% %각주4
  한국수학교육학회지 시리즈 B: <순수 및 응용 수학>에 투고한 논문
  중에는 종종 \AmS-\LaTeX 패키지를 쓰지 않고 만든 \TeX{} 파일들이 있다.
  주로
  \begin{quote}
  \texttt{\textbackslash documentstyle\{article\}}
  \end{quote}
  또는
  \begin{quote}
     \texttt{\textbackslash documentclass\{article\}}
  \end{quote}
  로 시작하는 경우인 데 편집실에서는 다른 여러 가지 사정 때문에 모두
  \begin{quote}
     \texttt{\textbackslash documentclass\{ksme-b\}}
  \end{quote}
  로 고친다. 이때 가장 애를 먹는 부분이 \env{definition} 환경과
  \env{remark}\,환경을 써야하는 선언적 문단의 본체이다. 투고자들이
  학술지 고유의 스타일에 맞추기 위하여 수작업으로 
  \texttt{\{\textbackslash rm $\cdots$\}}, \texttt{\{\textbackslash em
  $\cdots$\}}, {\tt \textbackslash emph\{$\cdots$\}}, 
  등의 명령어를 써서
  입력하여 두었는 데 이 때 \env{definition} 환경이나
  \env{remark} 환경을 쓰게 되면 
  \texttt{\{\textbackslash em
  $\cdots$\}}, {\tt \textbackslash emph\{$\cdots$\}}, 
  등의 명령어를 써서
  입력하였을 때는 로마체와 이탤릭체가 완전히 반전되지만
  \texttt{\{\textbackslash rm $\cdots$\}} 
  을 썼을 때는 반전이 되지 않는다.}
그래서 \textbf{Definition} 을 \env{definition} 환경에 따르고
\textit{Remark} 를 \env{remark} 환경에 따르게 하려면
% \begin{quote}
%   \hskip-13mm\textbf{예 3.}\hskip5mm\textbackslash {\tt
%     newtheorem\{thm\}\{Theorem\}}\\
%   \textbackslash \texttt{newtheorem\{lem\}[thm]\{Lemma\}}\\
%   \textbackslash \texttt{newtheorem\{prop\}[thm]\{Proposition\}}\\
%   \textbackslash \texttt{theoremstyle\{definition\}}\\
%   \textbackslash \texttt{newtheorem\{defn\}[thm]\{Definition\}}\\
%   \textbackslash \texttt{theoremstyle\{remark\}}\\
%   \textbackslash \texttt{newtheorem\{rem\}[thm]\{Remark\}}
% \end{quote}
\begin{quote}
\noindent\gethunindentlen{예 3}
\hspace*{-\hunindentlen}\usebox{\mybox}\hspace{4mm}\verb|\newtheorem{thm}{Theorem}| \\
   \verb|\newtheorem{thm}{Lemma}| \\
   \verb|\newtheorem{prop}{Proposition}| \\
   \verb|\theoremstyle{definition}|\\
   \verb|\newtheorem{defn}[thm]{Definition}| \\
   \verb|\theoremstyle{remark}| \\
   \verb|\newtheorem{rem}[thm]{Remark}| 
\end{quote}
와 같이 설정하면 된다. 예 3\,에서는 또 \textbf{Theorem}, \textbf{Lemma},
\textbf{Proposition}, \textbf{Definition} 과 \textit{Remark}\,를 모두
함께 일련 번호가 매겨지도록 설정하였다. 즉 \textbf{Theorem},
\textbf{Lemma}, \textbf{Proposition}, \textbf{Definition} 과
\textit{Remark}\,는 앞에서 부터 나타나는 순서대로 일련 번호가
매겨진다. 만약
% \begin{quote}
%   \hskip-13mm\textbf{예 4.}\hskip5mm\textbackslash {\tt
%     newtheorem\{thm\}\{Theorem\}}[section]\\
%   \textbackslash \texttt{newtheorem\{lem\}[thm]\{Lemma\}}\\
%   \textbackslash \texttt{newtheorem\{prop\}[thm]\{Proposition\}}\\
%   \textbackslash \texttt{newtheorem*\{cor\}\{Corollary\}}\\
%   \textbackslash \texttt{theoremstyle\{definition\}}\\
%   \textbackslash \texttt{newtheorem\{defn\}\{Definition\}}\\
%   \textbackslash \texttt{theoremstyle\{remark\}}\\
%   \textbackslash \texttt{newtheorem*\{rem\}\{Remark\}}
% \end{quote}
\begin{quote}
\noindent\gethunindentlen{예 4}
\hspace*{-\hunindentlen}\usebox{\mybox}\hspace{4mm}\verb|\newtheorem{thm}{Theorem}| \\
   \verb|\newtheorem{lem}[thm]{Lemma}| \\
   \verb|\newtheorem{prop}[thm]{Proposition}| \\
   \verb|\newtheorem*{cor}{Corollary}| \\
   \verb|\theoremstyle{definition}| \\
   \verb|\newtheorem{defn}{Definition}| \\
   \verb|\theoremstyle{remark}| \\
   \verb|\newtheorem*{rem}{Remark}|
\end{quote}
와 같이 설정하면 \textbf{Theorem}, \textbf{Lemma},
\textbf{Proposition} 는 절(Section)이 바뀔 때 마다 함께 새로 일련
번호가 시작되며 \textbf{Theorem 3.1}, \textbf{Lemma 3.2},
\textbf{Proposition 3.3}, $\cdots$ 처럼 번호가 붙지만
\textbf{Definition} 만은 절과 관계 없이 논문 전체로 앞에서 부터
\textbf{Definition 1}, \textbf{Definition~2}, $\cdots$ 로 번호가
붙는다. 한 편 \textbf{Corollary} 와 \textit{Remark} 는 번호가 붙지
않는다.

또 \textbf{Corollary}를 상황에 따라 하나만 있을 때는 \textbf{Corollary}
로 쓰고, 둘 이상 있으 때는 \textbf{Corollary 1}, \textbf{Corollary 2},
$\cdots$ 등으로 붙이고 싶으면 즉, 어떤 \textbf{Corollary}는 번호를 붙이고
어떤 \textbf{Corollary}는 번호를 붙이지 않으려면 이들을 모두 별도로
취급하여 아래 예 5와 같이 각각 별도의 환경으로 정하여 주어야한다.
% \begin{quote}
%   \hskip-13mm\textbf{예 5.}\hskip5mm\textbackslash {\tt
%     newtheorem*\{cor\}\{Corollary\}}\\
%   \textbackslash \texttt{newtheorem*\{cor1\}\{Corollary 1\}}\\
%   \textbackslash \texttt{newtheorem*\{cor2\}\{Corollary 2\}}\\
%   $\cdots$
% \end{quote}
\begin{quote}
\noindent\gethunindentlen{예 5}
\hspace*{-\hunindentlen}\usebox{\mybox}\hspace{4mm}\verb|\newtheorem*{cor}{Corollary}| \\
   \verb|\newtheorem*{cor1}{Corollary 1}| \\
   \verb|\newtheorem*{cor2}{Corollary 2}|
\end{quote}
또 만약
% \begin{quote}
%   \hskip-13mm\textbf{예 6.}\hskip5mm\textbackslash {\tt
%     newtheorem\{thm\}\{Theorem\}[section]}\\
%   \textbackslash \texttt{newtheorem\{cor\}\{Corollary\}[thm]}
% \end{quote}
\begin{quote}
\noindent\gethunindentlen{예 6}
\hspace*{-\hunindentlen}\usebox{\mybox}\hspace{4mm}\verb|\newtheorem{thm}{Theorem}| \\
   \verb|\newtheorem{cor}{Corollary}[thm]|
\end{quote}
와 같이 설정하면 \textbf{Theorem}은 절(Section)이 바뀔 때마다 새로
일련 번호가 시작되며 \textbf{Corollary}는 \textbf{Theorem}에 따라
번호가 붙는다. 가령 \textbf{Theorem 3.2}\,이면 그에 따른
\textbf{Corollary}는 \textbf{Corollary 3.2.1}, \textbf{Corollary
  3.2.2}, $\cdots$ 처럼 번호가 붙는다.

한편
% \begin{quote}
%   \hskip-13mm\textbf{예 7.}\hskip5mm\textbackslash {\tt
%     newtheorem*\{lema\}\{Lemma A\}}\\
%   \textbackslash \texttt{newtheorem*\{lemb\}\{Lemma B\}}\\
%   \textbackslash \texttt{newtheorem*\{KL\}\{Klein's Lemma\}}\\
%   \textbackslash \texttt{newtheorem*\{FLT\}\{Fermat's Last Theorem\}}
% \end{quote}
\begin{quote}
\noindent\gethunindentlen{예 7}
\hspace*{-\hunindentlen}\usebox{\mybox}\hspace{4mm}\verb|\newtheorem*{lema}{Lemma A}| \\
   \verb|\newtheorem*{lemb}{Lemma B}| \\
   \verb|\newtheorem*{KL}{Klein's Lemma}| \\
   \verb|\newtheorem*{FLT}{Fermat's Last Theorem}|
\end{quote}
와 같이 설정하면 \textbf{Lemma A}, \textbf{Lemma B}, \textbf{Klein's
  Lemma}, \textbf{Fermat's Last Theorem} 등으로 쓸 수 있다.

본문(text: 
% \textbackslash \texttt{maketitle} 
\verb|\maketitle|%
부터 
% \textbackslash {\tt bibliographystyle} 
\verb|\bibliographystyle|%
까지) 에서 \textbf{Definition}, \textbf{Theorem},
{\bf Lemma}, \textbf{Lemma A}, \textbf{Klein's Lemma},
\textbf{Fermat's Last Theorem}, \textbf{Proposition},
\textbf{Corollary}, \textbf{Corollary 1}, {\it Remark} 등의 입력은
% \begin{quote}
%   \textbackslash \texttt{begin\{defn\}\;\;$\cdots\;\;
%     \backslash$end\{defn\}}\\
%   \textbackslash \texttt{begin\{thm\}\;\;$\cdots\;\;
%     \backslash$end\{thm\}}\\
%   \textbackslash \texttt{begin\{lem\}\;\;$\cdots\;\;
%     \backslash$end\{lem\}}\\
%   \textbackslash \texttt{begin\{lema\}\;\;$\cdots\;\;
%     \backslash$end\{lema\}}\\
%   \textbackslash \texttt{begin\{KL\}\;\;$\cdots\;\;
%     \backslash$end\{KL\}}\\
%   \textbackslash \texttt{begin\{FLT\}\;\;$\cdots\;\;
%     \backslash$end\{FLT\}}\\
%   \textbackslash \texttt{begin\{prop\}\;\;$\cdots\;\;
%     \backslash$end\{prop\}}\\
%   \textbackslash \texttt{begin\{cor\}\;\;$\cdots\;\;
%     \backslash$end\{cor\}}\\
%   \textbackslash \texttt{begin\{cor1\}\;$\cdots\;\;
%     \backslash$end\{cor1\}}\\
%   \textbackslash \texttt{begin\{rem\}\;\;$\cdots\;\;
%     \backslash$end\{rem\}}
% \end{quote}
\begin{verbatim}
   \begin{defn} ...  \end{defn}
   \begin{thm}  ...  \end{thm}
   \begin{lem}  ...  \end{lem}
   \begin{lema} ...  \end{lema}
   \begin{KL}   ...  \end{KL}
   \begin{FLT}  ...  \end{FLT}
   \begin{prop} ...  \end{prop}
   \begin{cor}  ...  \end{cor}
   \begin{cor1} ...  \end{cor1}
   \begin{rem}  ...  \end{rem}
\end{verbatim}
등으로 하면 DVI 파일에서 원하는 결과가 나타난다.

다음은

\textbf{Deinition 1} (cf.~Lambert [9])\textbf{.}\;\;$\cdots$

\textbf{Theorem 1} (Chinese Remainder Theorem)\textbf{.}\;\;$\cdots$

\textbf{Lemma 2.1} (Mittelbach [9, p.~125])\textbf{.}\;\;$\cdots$

\noindent 와 같이 앞머리에 부수적인 설명을 넣어 보자. 이것은 비교적으로
간단하다.
% \textbackslash \texttt{begin\{defn\}[cf.\textbackslash ~Lambert
%   \textbackslash ref\{la\}] \;\;$\cdots\;\;\backslash$end\{defn\}}
% 
% \textbackslash \texttt{begin\{thm\}[Chinese Remainder Theorem]
%   \;\;$\cdots\;\;\backslash$end\{thm\}}
% 
% \textbackslash \texttt{begin\{lem\}[\{Mittelbach
%   \textbackslash ref[p.\textbackslash ~125]\{mi\}\}]
%   \;\;$\cdots\;\;\backslash$end\{lem\}}\\
\begin{verbatim}
   \begin{defn}[cf.\~Lambert \ref{la}] ... \end{defn}
   \begin{thm}[Chinese Remainder Theorem] ... \end{thm}
   \begin{lem}[Mittelbach \ref[p.~125]{mi}] ... \end{lem}
\end{verbatim}
등으로 하면 DVI 파일에서 원하는 결과가 나타난다. 위에서
\verb|\ref{la}|, \verb|\ref[p.~125]{mi}|
% \textbackslash ref\{la\}, \textbackslash ref[p.~125]\{mi\} 
등의 \{\;\} 속에
있는 la, mi 등은 ``key''라고 하는 것으로 \env{thebibbliography}
환경(\TeX{} 파일의 뒷쪽에 있거나 별도의 bbl 파일로 되어 있다.) 에서
% \begin{quote}
% \textbackslash \texttt{bibitem\{la\}}\\
% \textbackslash \texttt{bibitem\{mi\}}
% \end{quote}
\begin{verbatim}
   \bibitem{la}
   \bibitem{mi}
\end{verbatim}
등으로 입력되어 있다. 이번에는
\begin{quote}
     {\tt\bf 2.1 Definition.}\\
     {\tt\bf 2.2 Theorem.}\\
     {\tt\bf 2.3 Corollary.}
\end{quote}
등으로 번호가 앞쪽에 나가게 하여 보자

% \textbackslash \texttt{newtheorem} 
\verb|\newtheorem|
설정을 하기 전에 명령어
% \textbackslash \texttt{swapnumbers}
\verb|swapnumbers}|를 쓰면 된다.

% \begin{quote}
%   \hskip-13mm\textbf{예 8.}\hskip5mm\textbackslash \texttt{swapnumbers}\\
%   \textbackslash \texttt{theoremstyle\{definition\}[section]}\\
%   \textbackslash \texttt{newtheorem\{defn\}\{Definition\}}\\
%   \textbackslash \texttt{theoremstyle\{plain\}}\\
%   \textbackslash \texttt{newtheorem\{thm\}[defn]\{Theorem\}}\\
%   \textbackslash \texttt{newtheorem\{cor\}[defn]\{Corollary\}}
% \end{quote}
\begin{quote}
\noindent\gethunindentlen{예 8}
\hspace*{-\hunindentlen}\usebox{\mybox}\hspace{4mm}\verb|\swapnumbers| \\
   \verb|\theoremstyle{definition}[section]| \\
   \verb|\newtheorem{defn}{Definition}| \\
   \verb|\theoremstyle{plain}| \\
   \verb|\newtheorem{thm}[defn]{Theorem}| \\
   \verb|\newtheorem{cor}[defn]{Corollary}|
\end{quote}
와 같이 하면 원하는 결과가 DVI 파일에 나타난다.

\section{\textit{Proof}\, 환경}
\textit{Proof}\, 환경은 비교적으로 쉽다.
% \begin{quote}
%   \textbackslash \texttt{begin\{proof\}}$\cdots\backslash$\texttt{end\{proof\}}
% \end{quote}
\begin{verbatim}
   \begin{proof} ... \end{proof}
\end{verbatim}
식으로 하면 DVI 파일에서 \textit{Proof} 이란 앞머리는 이탤릭체로 되고
본체는 로마체로 되며, 증명 끝에 자동적으로 $\Box$\,이 붙는다.

만약 \textit{Proof of Theorem 1}로 하고 싶으면
% \begin{quote}
%   \textbackslash \texttt{begin\{proof\}[Proof of Theorem 1]}
%   $\cdots\backslash$\texttt{end\{proof\}}
% \end{quote}
\begin{verbatim}
   \begin{proof}[Proof of Theorem 1] ... \end{proof}
\end{verbatim}
으로 입력하면 된다. 이때 앞머리 \textit{Proof of Theorem 1} 이 모두
이탤릭체로 되는데
% \begin{quote}
% \textbackslash \{proof\}[Proof \{\textbackslash rm of\}
%   \{\textbackslash bf Theorem 1\}]
% \end{quote}
\begin{verbatim}
   \begin{proof}[Proof {\rm of} {\bf Theorem 1}]
\end{verbatim}
로 입력하면 \textit{Proof} of \textbf{Theorem 1} 과 같이
나타난다.

\AmS-\LaTeX 의 \textit{Proof} 환경에서는 증명이 끝나면 증명끝을
나타내는 $\Box$ 가 나타난다. 증명의 본체가 ``display 환경''에서 끝나지
않으면 증명 맨 끝줄의 오른쪽 끝에 $\Box$가 나타난다.  그런데 증명의
본체가 ``display 환경''에서 끝나면 $\Box$은 display 다음 줄의 오른쪽
끝에 나타난다. 이때 $\Box$를 display 줄에 놓고 싶으면 명령어
\textbackslash \texttt{qedhere}를 쓰면 된다. 그런데
% \begin{quote}
% \texttt{\$\$\; x\^\,2+y\^\,2=1.\textbackslash \texttt{quad}
%   \textbackslash \texttt{qedhere}\;\;\$\$}
% \end{quote}
\begin{verbatim}
   x^2 + y^2 = 1 \quad \qedhere
\end{verbatim}
로 입력하면
$$   x^2+y^2=1. \quad \Box   $$
과 같이 된다.  또
% \begin{quote}
% $\backslash${\tt begin\{equation*\}                                \\
%     \quad x\^\,2+y\^\,2=1.
%     $\backslash$tag*\{$\backslash$hspace\{-1em\}$\backslash$qedhere\}
%                                                                    \\
%     $\backslash$end\{equation*\} }
% \end{quote}
\begin{verbatim}
  \begin{equation*}
   \quad x^2 + y^2 = 1, \tag*{\hspace{-1em}\qedhere}
  \end{equation*}
\end{verbatim}
로 입력하면
\begin{equation*}
  \hskip60mm x^2+y^2=1.  \hskip58mm\Box
\end{equation*}
로 나타난다.

\section{정리류 환경을 쓰지 않았을 때의 문제점}
\AmS-\LaTeX 의 정리류 환경을 쓰지 않고 그냥 \LaTeX 의 ``정리 환경''을
쓰면 모든 선언적 문단은 앞머리를 볼드체로 하고, 본체를 이탤릭체로 하여
DVI 파일이 만들어진다. 그래서 많은 논문의 저자들은 본체에 수작업으로
\begin{quote}
  \{\textbackslash \texttt{rm\; $\cdots$\}, \quad \{\textbackslash em\;
    $\cdots$\}, \quad \textbackslash emph\{\;$\cdots$\}}
\end{quote}
등을 써서 이탤릭체를 모두 로마체로 바꾸고 있다. 그리고 이탤릭체가
되어야하는 곳은 정작 아무 표시를 하지 않는다. 편집실에서는 다른 많은
이유 때문에 \TeX 파일의 첫머리에 나타나는
% \begin{quote}
%      \texttt{\textbackslash documentclass\{$\cdots$\}}
% \end{quote}
\begin{verbatim}
   \documentclass{...}
\end{verbatim}
의 $\cdots$를 고유의 클래스로 바꾼다. 이때 가장 애를 먹는 부분이
\env{definition} 환경과 \env{remark} 환경을 써야 하는 선언적
문단의 본체이다.명령어 
% \{\textbackslash \texttt{em\; $\cdots$\}},
% \textbackslash \texttt{emph\{$\cdots$\}}
\verb|{\em ...}|, \verb|\emph{...}|을 썼을 때는 로마체와 이탤릭체가
완전히 반전되지만 명령어 
% \{\textbackslash \texttt{rm\; $\cdots$\}}
\verb|{\rm ...}|을 썼을
때는 반전이 이루어지지 않는다. 또 로마체로 남아 있는 부분은
% \textbackslash \texttt{emph\{$\cdots$\}} 
\verb|\emph{...}|
등을 써서 이탤릭체로 바꾸어 주어야 한다.

이 때문에 {\AmS}-{\LaTeX}\,의 정리류 환경을 불러 오는 클래스를 애초부터
쓰는 것이 가장 안전한 방법이다. 어떤 학회지의 클래스가
{\AmS}-{\LaTeX}\, 패키지(정확히는 \amsthmstyle )를 불러 오는지 아닌지를
알고 싶으면 그 클래스를 쓴 \TeX 을 한 번 컴파일하고 AUX 파일을 보면 알
수 있다. 만약 AUX 파일에서 \amsthmstyle 를 불러 온 흔적이 없으면
Preamble 에
\begin{verbatim}
    \usepackage{amsmath}
\end{verbatim}
를 입력하여 \amsthmstyle 를 불러 들이면 {\AmS}-{\LaTeX}의 정리류 환경을
쓸 수 있다.

%\newpage
\bibliographystyle{amsplain}
\begin{thebibliography}{99}
%\baselineskip 5mm
\bibitem{ko}
고기형: \emph{한글과 \TeX}. 청문각, 서울, 1995.

\bibitem{ch01}
최영한: 순수 및 응용 수학 논문의 참고 문헌 작성 요령.
  \emph{한국수학교육학회지 시리즈 E 수학교육 논문집} \textbf{11} (2001),
  1--25.

\bibitem{ch02}
{\bysame}: \emph{수학 논문의 정리류(Theorem-like)를 쓰는 요령}.
  To Appear.

\bibitem{ch03}
{\bysame}: \emph{Non-local \LaTeX 의 설치와 수학 논문의 참고 문헌의
  작성 요령}. To Appear.

\bibitem{amg}
  Amer.~Math.~Soc.: \emph{User's Guide for the \texttt{amsmath}
  Package} (Version 2.0). {Amer. Math.~Soc.}, Providence, Rhode
  Island, 1999. \newline
  [이 문헌은 {\tt
  c:\textbackslash texmf\textbackslash doc\textbackslash latex\textbackslash amslatex}
  디렉터리에서 \texttt{amsldoc.dvi} 파일을 찾아 인쇄하면 된다.]

\bibitem{ami}
\bysame: \emph{Instruction for Preparation of Papers and
  Monographs \AmS-\LaTeX}. {Amer.\ Math.\ Soc.}, Providence,
  RI, 2000. \newline
  [이 문헌은 {\tt
  c:\textbackslash texmf\textbackslash doc\textbackslash latex\textbackslash amslatex}
  디렉터리에서 \texttt{instr-l.dvi} 파일을 찾아 인쇄하면 된다.]

\bibitem{amt}
\bysame: \emph{Using the \texttt{amsthm} Package}, Version 2.07.
  {Amer.\ Math.\ Soc.}, Providence, RI, 2000. \newline
  [이 문헌은 {\tt
  c:\textbackslash texmf\textbackslash doc\textbackslash latex\textbackslash amslatex}
  디렉터리에서 \texttt{amsthdoc.dvi} 파일을 찾아 인쇄하면 된다.]

\bibitem{go}
  Michel~Goosens, Frank~Mittelbach, and Alexander Samarin: \emph{The
  \LaTeX~Companion}. {Ad\-di\-son-Wes\-ley}, Reading, MA, 1994.

\bibitem{kop}
  Helmut Kopka and Patrick W. Daly: \emph{A Guide to \LaTeX},
  3rd Ed. {Ad\-di\-son-Wes\-ley}, Reading, MA, 1999.

\bibitem{la}
  Leslie Lambert: \emph{\LaTeX}. {Ad\-di\-son-Wes\-ley}, Reading,
  MA, 1994.

\bibitem{mi}
  Frank~Mittelbach: \emph{An extenssion of the \LaTeX~theorem
  environment}. 2000. \newline
  [이 문헌은 {\tt
  c:\textbackslash texmf\textbackslash doc\textbackslash latex\textbackslash tools}
  디렉터리에서 \texttt{theorem.dvi} 파일을 찾아 인쇄하면 된다.]
\end{thebibliography}

\end{document}

%%
% end of file.
%%

    \begin{abstract}
    We present a new procedural method for photorealistic rendering of woven fabric material.
%    Realistic rendering of woven fabric is still a hard problem in graphics community.
%    Several methods have been proposed to render virtual fabrics with procedural techniques.
%    However, they still have various limitations in producing realistic images.
%    In order to enhance the realism, researchers employed example-based approaches which utilize measured data of surface reflectance properties.
%    Although the example-based approaches drastically enhance the realism of virtual fabric rendering,
%    those methods have serious disadvantage that they require huge amount of storage for the various reflectance properties of diverse materials.
    The goal of our research is to provide a new procedural method that renders photorealistic woven fabric without any measured data.
    The proposed method models the reflectance properties of woven fabric with alternating anisotropy and yarn-level surface normal manipulation.
    The proposed method models the reflectance properties of woven
    fabric with alternating anisotropy and deformed microfacet
    distribution function (MDF). The deformed MDF effectively represents the
    perturbed reflectance caused by bumps on the woven fabric surface.
    The experimental results show the proposed method can be successfully applied to photorealistic rendering of diverse woven fabric materials.
    \end{abstract}
